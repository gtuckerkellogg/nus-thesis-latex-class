% \iffalse meta-comment -------------------------------------------------------
%   -------------------------------------------------------------------------
%   Copyright 2024 Greg Tucker-Kellogg 
%   
%   This work is licensed under a Creative Commons Attribution-ShareAlike 4.0
%   International License (https://creativecommons.org/licenses/by-sa/4.0/).
%   -------------------------------------------------------------------------
% \fi
% \iffalse
%   
% 
%<cls>\NeedsTeXFormat{LaTeX2e}[2022-06-01]
%<cls>\RequirePackage{l3keys2e}
%<cls>\ProvidesExplClass{nus-thesis}{2024-00-00}{0.0.0.a}{}{NUS thesis class}
% 
%<*driver>
\documentclass{l3doc}
\usepackage{framed}
\EnableCrossrefs
\CodelineIndex
\RecordChanges
\begin{document}
\DocInput{\jobname.dtx}
\end{document}
%</driver> 
% \fi
% 
% \GetFileInfo{nus-thesis.dtx}
% \title{The \textsf{nus-thesis} class\thanks{This document
%   corresponds to \textsf{⟨package⟩}~\fileversion,
%   dated~\filedate.}}
% \author{Greg Tucker-Kellogg \\ \texttt{dbsgtk@nus.edu.sg}}
% \maketitle
% \begin{abstract}
%   This is the implementation of the NUS thesis example.
% \end{abstract}
% 
% \tableofcontents
% 
% \section{Introduction}
% 
% \begin{function}{\pdftexengine, \xetexengine, \luatexengine}
%   \begin{syntax}
%     \cmd{\pdftexengine}
%     \cmd{\xetexengine}
%     \cmd{\luatexengine}
%   \end{syntax}
%   These commands alias the built-in \LaTeX3{} macros
%   \cs{sys_if_engine_pdftex_p:},
%   \cs{sys_if_engine_xetex_p:}, and
%   \cs{sys_if_engine_luatex_p:}.
%   They can be used to check which engine the user is currently running.
% \end{function}
% 
% 
% 
% \section{Class options}
% Class options can be set with the \cs{documentclass}, later as keys, or in some cases in dedicated commands.
%
%
% \DescribeOption{openright}
% This option provides double-sided printing and chapters (or equivalent) opening on the right side page.
%
% \DescribeOption{document-type}
% thesis, report
%
% \DescribeOption{candidacy}
% degree candidacy
%
%    \begin{macrocode}
%<*cls>      

\str_const:Nn \c__classname_base_class_str { book }

% Make sure a few of these base class options are set properly
\PassOptionsToClass{a4paper}{\c__classname_base_class_str}
\PassOptionsToClass{12pt}{\c__classname_base_class_str}

\keys_define:nn { nus-thesis }
{
%    \end{macrocode}
% \begin{function}{openright}
% \end{function}
% \begin{macrocode}
  , openright .bool_set:N = \l__classname_openright_bool
  , openright .initial:n = { false }
  , author .tl_set:N = \l__thesis_author_tl
  , author .code:n = { \author{#1} }
  , author .initial:n = { (author~undefined)  }
  , title .tl_set:N = \l__thesis_title_tl  
  , title .code:n = { \title{#1} }
  , title .initial:n = { (title~undefined) }  
  , university .tl_set:N = \l__thesis_university_tl
  , university .initial:n = { National~University~of~Singapore }
  , department .tl_set:N = \l__thesis_department_tl
  , department .initial:n = { (department) }
  , document-type .tl_set:N = \l__documenttype_tl
  , document-type .initial:n = { thesis }
  , candidacy .tl_set:N = \l__thesis_candidacy_tl
  , candidacy .initial:n = { (candidacy) }
  , submission-year .tl_set:N = \l__thesis_year_tl
  , submission-year .initial:n =  { \number\year }
  , qualification .tl_set:N = \l__thesis_qualification_tl
  , qualification .initial:n = { (qualification undefined)  }
  , qualification .code:n = {\qualification{#1}}
  % \DeclareDefaultOption
  , unknown .code:n =
  {
    \iow_term:x
    {
      Passing~option~\CurrentOption \c_space_tl to~
      \str_use:N \c__classname_base_class_str
    }
    \PassOptionsToClass { \CurrentOption } { \c__classname_base_class_str }
  }
}

\ProcessKeysOptions { nus-thesis }

\bool_if:NTF \l__classname_openright_bool
{
  \PassOptionsToClass{openright}{\c__classname_base_class_str}
}{
  \PassOptionsToClass{oneside}{\c__classname_base_class_str}
}

\LoadClass { \c__classname_base_class_str }


\DeclareDocumentCommand \ThesisSetup { m } {
  \keys_set:nn { nus-thesis } { #1 }
}

\RequirePackage[UKenglish]{isodate}
\RequirePackage{color}
\RequirePackage{iftex}
\RequirePackage[svgnames]{xcolor}
\RequirePackage[]{graphicx}
\RequirePackage{epstopdf}
\RequirePackage{setspace}
\RequirePackage{titlesec}
\RequirePackage{ifthen}
\RequirePackage[top=1in, bottom=1in, left=1.5in, right=1.5in]{geometry}
\RequirePackage{amsmath}
\RequirePackage{lipsum}
\RequirePackage{subfig}
\RequirePackage{longtable}
\RequirePackage{booktabs}
\RequirePackage{mathtools}
\RequirePackage{multirow}
\RequirePackage{multicol}
\RequirePackage{fancyvrb}
\RequirePackage{pbox}
\RequirePackage{pdflscape}
\RequirePackage{pstricks}
\RequirePackage{minitoc}
\RequirePackage{acronym}
\RequirePackage[section]{placeins}
\RequirePackage[nottoc]{tocbibind}
\RequirePackage[titletoc]{appendix}
\RequirePackage{mdwtab}
\RequirePackage{caption}
\RequirePackage{floatrow}
\RequirePackage[tracking=true]{microtype}

\ifPDFTeX%
  \RequirePackage[T1]{fontenc}
  \RequirePackage[utf8]{inputenc}
  \RequirePackage{textcomp} % provides euro and other symbols
\else % if luatex or xelatex
  \RequirePackage{fontspec}
  \RequirePackage{xunicode}
\fi

\RequirePackage{hyperref}
\hypersetup{
  linkcolor = DarkBlue,
  citecolor = DarkBlue,
  plainpages = false,
  hyperfootnotes = false,
  hypertexnames = false,
  allbordercolors = white,
}
\RequirePackage[nameinlink,noabbrev, capitalise]{cleveref}

%%%%%%%%%%%%%%%%%%%%%%%%%%%%%%%%%%%%%%%%%%%%%%%%%%%%%%%%%% 
% Document Settings
%%%%%%%%%%%%%%%%%%%%%%%%%%%%%%%%%%%%%%%%%%%%%%%%%%%%%%%%%% 
% Double Line Spacing
\doublespacing

% Plain Page Style
\pagestyle{plain}

% Labels Declaration
\graphicspath{ {./images/} }

%%%%%%%%%%%%%%%%%%%%%%%%%%%%%%%%%%%%%%%%%%%%%%%%%%%%%%%%%% 
% New Variables
%%%%%%%%%%%%%%%%%%%%%%%%%%%%%%%%%%%%%%%%%%%%%%%%%%%%%%%%%% 
% Defaults of Variables
\gdef\@subtitle{\space}
\gdef\@qualification{\relax}
\gdef\@keywords{\space}
\gdef\nus@abbrev{\space Test \space MAC }

% Define Inputs
\def\supervisor#1{\gdef\@supervisor{#1}}
\def\subtitle#1{\gdef\@subtitle{#1}}
\def\qualification#1{\gdef\@qualification{#1}}
\def\keywords#1{\gdef\@keywords{#1}}


\newcommand*{\textsomels}[1]{\textls[50]{#1}}


%%%%%%%%%%%%%%%%%%%%%%%%%%%%%%%%%%%%%%%%%%%%%%%%%%%%%%%%%% 
% Renewed Commands
%%%%%%%%%%%%%%%%%%%%%%%%%%%%%%%%%%%%%%%%%%%%%%%%%%%%%%%%%% 
% Title Page
\renewcommand{\titlepage}{
  \thispagestyle{empty}
  \begin{center}\hyphenpenalty=10000
    \Large\bfseries\uppercase\expandafter{\@title}\\
    \normalsize\mdseries\uppercase\expandafter{\@subtitle}
    \vfill%

  \end{center}\vfill
  \titlepagebody
  \vfill
  \begin{center}\hyphenpenalty=10000
    \titlepagefooter
  \end{center}\newpage}

% Title (Cover and Title)
\renewcommand{\maketitle}{
  \coverpage
  \titlepage
}

% Table captions
\renewcommand{\tablename}{Table}
\DeclareCaptionFormat{hfillstart}{#1#2#3}
\DeclareFloatFont{normalsize}{\normalsize}
\captionsetup[table]{
  format=hfillstart,
  labelsep=newline,
  justification=centering,
}

% Figure caption
\renewcommand{\figurename}{Fig.}
\captionsetup[figure]{
  labelsep=period,
}

\floatsetup[table]{capposition=top, font=normalsize}

%%%%%%%%%%%%%%%%%%%%%%%%%%%%%%%%%%%%%%%%%%%%%%%%%%%%%%%%%% 
% New Commands
%%%%%%%%%%%%%%%%%%%%%%%%%%%%%%%%%%%%%%%%%%%%%%%%%%%%%%%%%% 
% Coverpage
\newcommand{\coverpage}{
  \thispagestyle{empty}\vspace*{0.3in}
  \begin{center}\hyphenpenalty=10000
    \Large\bfseries\uppercase\expandafter{\@title}\\
    \Large\mdseries\uppercase\expandafter{\@subtitle}
  \end{center}\vspace*{8cm}
  \begin{center}\hyphenpenalty=10000
    \large\bfseries\uppercase\expandafter{\@author}
  \end{center}\vfill
  \begin{center}\hyphenpenalty=10000
    \normalsize\bfseries
    \large\uppercase\expandafter{\l__thesis_university_tl}\vspace*{0.2cm} \\
    \large\uppercase\expandafter{\l__thesis_year_tl}
  \end{center}
  \newpage}

% Title Page Body
\@ifundefined{titlepagebody}{
  \newcommand{\titlepagebody}{
    \begin{center}\hyphenpenalty=10000
      \large\uppercase\expandafter{{\bfseries\@author}}\\ \textit{\expandafter{\@qualification}}
    \end{center}}}{}

% Title Page Footer
\@ifundefined{titlepagefooter}{
  \newcommand{\titlepagefooter}{\large\bfseries\uppercase\expandafter{
      \textsomels{A~\uppercase\expandafter{\l__documenttype_tl}~submitted~
        for~the~Degree~of~\\\uppercase\expandafter{\l__thesis_candidacy_tl}\\ Department~of~ \uppercase\expandafter{\l__thesis_department_tl}\\ \uppercase\expandafter{\l__thesis_university_tl}}\\ 
      \vspace*{2cm}\l__thesis_year_tl\\
    }}}{}

% Declaration Page
\@ifundefined{declarationpage}{
  \newcommand{\declarationpage}{
    \thispagestyle{empty}
    \begin{center}\hyphenpenalty=10000
      \large\bfseries\uppercase\expandafter{DECLARATION}
    \end{center}
    \vspace*{1cm}
    \begin{center}\hyphenpenalty=10000
      I~hereby~declare~that~the~thesis~is~my~original\\
      work~and~it~has~been~written~by~me~in~its~entirety.\\
      I~have~duly~acknowledged~all~the~sources~of\\
      information~which~have~been~used~in~the~thesis.\\
      \vspace*{0.5cm}
      This~thesis~has~also~not~been~submitted~for~any\\
      degree~in~any~university~previously.\\
      \vspace*{2cm}
      % \begin{figure}[!h]
      %   \subfloat{ \includegraphics[width=0.3\textwidth]{signature} }
      % \end{figure}
      \vspace*{-1cm}
      \begin{center}
        \rule{8cm}{0.5pt}
      \end{center}
      \vspace*{-0.5cm}
      \@author \\
      \today
    \end{center}\newpage}} {}



% \titleformat{\chapter}[display]{
% \bfseries\Huge
% }{
%   \filleft\MakeUppercase{\chaptertitlename} \Huge\thechapter}
%   {4ex}{\titlerule%
%   \vspace{2ex}%
%   \filright}[\vspace{2ex}\titlerule]%


\titleformat{\chapter}[display]{
  \bfseries\Huge
}{\filleft\MakeUppercase{\textsomels{\chaptertitlename}\  \thechapter}}{1ex}{}[]
\titlespacing*{\chapter}{0pt}{*4}{*4}

% Chapfont
% \newcommand{\chapfont}[1]{
% \noindent\textbf{\begin{Huge}#1\end{Huge}}\vspace*{1cm}\\}

% Acknowledgment
\newcommand{\acknowledgment}[1]{
  \vspace*{2.75cm}
% \chapfont{Acknowledgments}#1\vfill
\newpage}

% Abstract
% \newcommand{\abstract}[1]{
% \vspace*{2.75cm}
% \chapfont{Summary}#1\vfill
% \begin{flushright}\today\end{flushright}\newpage}

\newcommand{\abstract}[1]{
  \setcounter{secnumdepth}{-2}
  \chapter{Abstract}
  #1\vfill
  \setcounter{secnumdepth}{2}
}

% Strict Compliant
\@ifundefined{@nusstrict}{}{
  \chapternumberfont{\centering\mdseries\uppercase}
  \chaptertitlefont{\centering\mdseries\uppercase}
  \sectionfont{\raggedright}
  \renewcommand{\chaptername}{CHAPTER}
  \renewcommand{\appendixname}{APPENDIX}
% \renewcommand{\chapfont}[1]{{\centering\mdseries\begin{Huge}#1\end{Huge}\vspace*{1cm}\\}}
}

% Format text referring to appendices
\crefname{appsec}{Appendix}{Appendices}
\crefname{figure}{Fig.}{Fig.}
\crefname{table}{Table}{Tables}
\crefname{equation}{}{}

% Table float box with bottom caption, box width adjusted to content
\newfloatcommand{capbtabbox}{table}[][\FBwidth]

%%%%%%%%%%%%%%%%%%%%%%%%%%%%%%%%%%%%%%%%%%%%%%%%%%%%%%%%%% 
% Adapted from losymbol.sty
%%%%%%%%%%%%%%%%%%%%%%%%%%%%%%%%%%%%%%%%%%%%%%%%%%%%%%%%%% 
% losymbol.sty by gerry van dooren, may 1992, revised august 1992.
% elecgd@urc.tue.nl
% 
% definition is nearly analogous to the setup for list of figures
% and list of tables; use same definition using \@starttoc,
% however default extension is now .los
% 
\def\losymbolsname{List~of~Symbols}
\def\loabbrevname{List~of~Abbreviations}
\def\listofsymbolsnabbrev{
  \chapter*{\losymbolsname \@mkboth
    {\uppercase{\losymbolsname}}{\uppercase{\losymbolsname}}}
  \@starttoc{los}
  \chapter*{\loabbrevname \@mkboth
    {\uppercase{\loabbrevname}}{\uppercase{\loabbrevname}}}
  \begin{acronym}[RADAR12345]
    \@starttoc{loab}
  \end{acronym}
}
% 
% command for putting text in file \jobname.loa is \addsymbol
% command that is used for actually producing text is called
% \makenicesymbolline
\def\addsymbol#1#2{%
  \addtocontents{los}{\protect \makenicesymbolline{#1}{#2}}}
\def\addabbrev#1#2{%
  \addtocontents{loab}{\protect \makeniceabbrevline{#1}{#2}}}
% 
% physical dimensions of columns
\newdimen\@lcsym %width of left column
\@lcsym=2cm      %default 3cm
% 
\newdimen\@mcsym %width of mid column (space)
\@mcsym=1cm      %default 2cm
% 
\newdimen\@rcsym %width of right column
\@rcsym=\hsize   %initial width=textwidth
% 
\advance\@rcsym by -\@lcsym %subtract lcsym
\advance\@rcsym by -\@mcsym %subtract mid
% 
% command for producing the actual line uses standard minipages
% to cope with long definitions
% 
\def\makenicesymbolline#1#2{%
  \begin{minipage}[t]{\@lcsym}#1\end{minipage}%
  $\hspace{\@mcsym}$%
  \begin{minipage}[t]{\@rcsym}#2\end{minipage}\newline}
\def\makeniceabbrevline#1#2{%
  % \begin{minipage}[t]{\@lcsym}#1\end{minipage}%
  % $\hspace{\@mcsym}$%
  % \begin{minipage}[t]{\@rcsym}#2\end{minipage}\newline
  \acro{#1}{#2}}
%</cls>% 
%    \end{macrocode}
% 
% \iffalse

% \fi
% \Finale
\endinput
